\definecolor{bg}{rgb}{0.97,0.97,0.98}

\section*{Appendix}
main.rs
\begin{minted}[fontsize=\footnotesize, bgcolor=bg]{rust}
pub mod activator;
pub mod cross;
pub mod flood;
pub mod loss;
pub mod model;
pub mod utills;

use std::error::Error;

fn main() -> Result<(), Box<dyn Error>> {
    // training code
    flood::flood_8_4_1(0.01, 0.01, "flood-8-4-1", true)?; // base
    flood::flood_8_4_1(0.01, 0.0, "flood-8-4-1_2", true)?; // no momentum
    flood::flood_8_4_1(0.0001, 0.01, "flood-8-4-1_3", true)?; // small learning rate
    flood::flood_8_4_1(0.01, 0.01, "flood-8-4-1_4", false)?; // no data preprocessing
    flood::flood_8_8_1(0.01, 0.01, "flood-8-8-1")?; 

    cross::cross_2_4_1(0.01, 0.01, "cross-2-4-1")?; // base
    cross::cross_2_4_1(0.01, 0.0, "cross-2-4-1_2")?; // no momentum
    cross::cross_2_4_1(0.0001, 0.01, "cross-2-4-1_3")?; // small learning rate
    cross::cross_2_8_1(0.01, 0.01, "cross-2-8-1")?;

    Ok(())
}
\end{minted}
\noindent flood/mod.rs
\begin{minted}[fontsize=\footnotesize, bgcolor=bg]{rust}
    //! Contains training code for variations of flood dataset models.
    use super::activator;
    use super::loss;
    use super::model;
    use super::utills;
    
    use model::{Layer, Net};
    use std::error::Error;
    use std::fs;
    use std::io::Write;
    use std::time::{Duration, Instant};
    use utills::data;
    use utills::graph;
    use utills::io;
    
    pub fn flood_8_4_1(
        lr: f64,
        momentum: f64,
        folder: &str,
        standardize: bool,
    ) -> Result<(), Box<dyn Error>> {
        fn model() -> Net {
            let mut layers: Vec<model::Layer> = vec![];
            layers.push(Layer::new(8, 4, 1.0, activator::sigmoid()));
            layers.push(Layer::new(4, 1, 1.0, activator::linear()));
            Net::from_layers(layers)
        }
    
        flood_fit(&model, lr, momentum, folder, standardize)?;
        Ok(())
    }
    
    pub fn flood_8_8_1(lr: f64, momentum: f64, folder: &str) -> Result<(), Box<dyn Error>> {
        fn model() -> Net {
            let mut layers: Vec<model::Layer> = vec![];
            layers.push(Layer::new(8, 8, 1.0, activator::sigmoid()));
            layers.push(Layer::new(8, 1, 1.0, activator::linear()));
            Net::from_layers(layers)
        }
    
        flood_fit(&model, lr, momentum, folder, true)?;
        Ok(())
    }
    
    fn mse_to_rmse(mse: &Vec<f64>) -> Vec<f64> {
        mse.iter().map(|v| v.sqrt()).collect()
    }
    
    pub fn flood_fit(
        model: &dyn Fn() -> Net,
        lr: f64,
        momentum: f64,
        folder: &str,
        standardize: bool,
    ) -> Result<(), Box<dyn Error>> {
        let (models, img) = utills::io::check_dir(folder)?;
    
        let dataset = data::flood_dataset()?;
        let mut loss = loss::Loss::mse();
        let epochs = 1000;
    
        let mut cv_valid_loss: Vec<f64> = vec![];
        let mut cv_train_loss: Vec<f64> = vec![];
    
        let mut r2_score: Vec<f64> = vec![];
        let mut loss_g = graph::LossGraph::new();
        let start = Instant::now();
        for (j, dt) in dataset.cross_valid_set(0.1).iter().enumerate() {
            // creating a model
            let mut net = model();
    
            // get training set and validation set
            let training_set = if standardize {
                data::standardization(&dt.0, dt.0.mean(), dt.0.std())
            } else {
                dt.0.clone()
            };
    
            let validation_set = if standardize {
                data::standardization(&dt.1, dt.0.mean(), dt.0.std())
            } else {
                dt.1.clone()
            };
            //let training_set = data::minmax_norm(&dt.0, dt.0.min(), dt.0.max());
            //let validation_set = data::minmax_norm(&dt.1, dt.1.min(), dt.1.max());
    
            // training
            let mut loss_vec: Vec<f64> = vec![];
            let mut valid_loss_vec: Vec<f64> = vec![];
            for i in 0..epochs {
                let mut running_loss: f64 = 0.0;
    
                for data in training_set.get_shuffled() {
                    let result = net.forward(&data.inputs);
    
                    running_loss += loss.criterion(&result, &data.labels);
                    loss.backward(&mut net.layers);
    
                    net.update(lr, momentum);
                }
                running_loss /= training_set.len() as f64;
                loss_vec.push(running_loss);
    
                let mut valid_loss: f64 = 0.0;
                for data in validation_set.get_datas() {
                    let result = net.forward(&data.inputs);
                    valid_loss += loss.criterion(&result, &data.labels);
                }
                valid_loss /= validation_set.get_datas().len() as f64;
                valid_loss_vec.push(valid_loss);
    
                if i == epochs - 1 {
                    // log score
                    let label_mean = validation_set.get_datas().iter().fold(0f64, |mean, val| {
                        mean + val.labels[0] / validation_set.len() as f64
                    });
    
                    let mut total_sum_sqr = 0f64;
                    let mut sum_sqr = 0f64;
    
                    for data in validation_set.get_datas() {
                        let result = net.forward(&data.inputs);
                        sum_sqr += (data.labels[0] - result[0]).powi(2);
                        total_sum_sqr += (data.labels[0] - label_mean).powi(2);
                    }
    
                    r2_score.push(1.0 - (sum_sqr / total_sum_sqr));
                    cv_valid_loss.push(valid_loss);
                    cv_train_loss.push(running_loss);
                }
    
                println!(
                    "iteration: {}, epoch: {}, loss: {:.6}, valid_loss: {:.6}",
                    j, i, running_loss, valid_loss
                );
            }
    
            loss_g.add_loss(loss_vec, valid_loss_vec);
            io::save(&net.layers, format!("{}/{}.json", models, j))?;
        }
        let duration: Duration = start.elapsed();
    
        let mut file = fs::File::create(format!("{}/result.txt", models))?;
        file.write_all(
            format!(
                "cv_score: {:?}\n\nr2_score: {:?}\n\ntime used: {:?}",
                cv_valid_loss, r2_score, duration
            )
            .as_bytes(),
        )?;
    
        loss_g.draw(format!("{}/loss.png", img))?;
    
        graph::draw_2hist(
            [mse_to_rmse(&cv_valid_loss), mse_to_rmse(&cv_train_loss)],
            "Validation/Training RMSE",
            ("Iterations", "Validataion/Training RMSE"),
            format!("{}/cv_l.png", img),
        )?;
    
        graph::draw_histogram(
            r2_score,
            "Cross Validation R2 Scores",
            ("Iterations", "R2 Scores"),
            format!("{}/r2_score.png", img),
        )?;
        Ok(())
    }
\end{minted}
\noindent cross/mod.rs
\begin{minted}[fontsize=\footnotesize, bgcolor=bg]{rust}
//! Contains training code for variations of cross.pat dataset models.
use super::activator;
use super::loss;
use super::model;
use super::utills;

use model::{Layer, Net};
use std::error::Error;
use std::fs;
use std::io::Write;
use std::time::{Duration, Instant};
use utills::data;
use utills::graph;
use utills::io;

fn confusion_count(matrix: &mut [[i32; 2]; 2], result: &Vec<f64>, label: &Vec<f64>) {
    let threshold = 0.5;
    if result[0] > threshold {
        // true positive
        if label[0] == 1.0 {
            matrix[0][0] += 1
        } else {
            // false negative
            matrix[1][0] += 1
        }
    } else if result[0] <= threshold {
        // true negative
        if label[0] == 0.0 {
            matrix[1][1] += 1
        }
        // false positive
        else {
            matrix[0][1] += 1
        }
    }
}

pub fn cross_2_4_1(lr: f64, momentum: f64, folder: &str) -> Result<(), Box<dyn Error>> {
    fn model() -> Net {
        let mut layers: Vec<model::Layer> = vec![];
        layers.push(Layer::new(2, 4, 1.0, activator::sigmoid()));
        layers.push(Layer::new(4, 1, 1.0, activator::sigmoid()));
        Net::from_layers(layers)
    }

    cross_fit(&model, lr, momentum, folder)?;
    Ok(())
}

pub fn cross_2_8_1(lr: f64, momentum: f64, folder: &str) -> Result<(), Box<dyn Error>> {
    fn model() -> Net {
        let mut layers: Vec<model::Layer> = vec![];
        layers.push(Layer::new(2, 8, 1.0, activator::sigmoid()));
        layers.push(Layer::new(8, 1, 1.0, activator::sigmoid()));
        Net::from_layers(layers)
    }

    cross_fit(&model, lr, momentum, folder)?;
    Ok(())
}

pub fn cross_fit(
    model: &dyn Fn() -> Net,
    lr: f64,
    momentum: f64,
    folder: &str,
) -> Result<(), Box<dyn Error>> {
    let (models, img) = utills::io::check_dir(folder)?;

    let dataset = data::cross_dataset()?;
    let mut loss = loss::Loss::mse();
    let epochs = 7500;

    let mut valid_acc: Vec<f64> = vec![];
    let mut train_acc: Vec<f64> = vec![];
    let mut loss_g = graph::LossGraph::new();
    let mut matrix_vec: Vec<[[i32; 2]; 2]> = vec![];

    let start = Instant::now();
    for (j, dt) in dataset.cross_valid_set(0.1).iter().enumerate() {
        // creating a model
        let mut net = model();

        // get training set and validation set
        let training_set = &dt.0;
        let validation_set = &dt.1;

        // training
        let mut loss_vec: Vec<f64> = vec![];
        let mut valid_loss_vec: Vec<f64> = vec![];
        for i in 0..epochs {
            let mut running_loss: f64 = 0.0;

            for data in training_set.get_shuffled() {
                let result = net.forward(&data.inputs);

                running_loss += loss.criterion(&result, &data.labels);
                loss.backward(&mut net.layers);

                net.update(lr, momentum);
            }
            running_loss /= training_set.len() as f64;
            loss_vec.push(running_loss);

            let mut valid_loss: f64 = 0.0;
            for data in validation_set.get_datas() {
                let result = net.forward(&data.inputs);
                valid_loss += loss.criterion(&result, &data.labels);
            }
            valid_loss /= validation_set.get_datas().len() as f64;
            valid_loss_vec.push(valid_loss);

            if i == epochs - 1 {
                let mut matrix = [[0, 0], [0, 0]];
                for data in validation_set.get_datas() {
                    let result = net.forward(&data.inputs);
                    confusion_count(&mut matrix, &result, &data.labels);
                }

                let mut matrix2 = [[0, 0], [0, 0]];
                for data in training_set.get_datas() {
                    let result = net.forward(&data.inputs);
                    confusion_count(&mut matrix2, &result, &data.labels);
                }
                valid_acc.push((matrix[0][0] + matrix[1][1]) as f64 / validation_set.len() as f64);
                train_acc.push((matrix2[0][0] + matrix2[1][1]) as f64 / training_set.len() as f64);
                matrix_vec.push(matrix);
            }

            println!(
                "iteration: {}, epoch: {}, loss: {:.6}, valid_loss: {:.6}",
                j, i, running_loss, valid_loss
            );
        }

        loss_g.add_loss(loss_vec, valid_loss_vec);
        io::save(&net.layers, format!("{}/{}.json", models, j))?;
    }
    let duration: Duration = start.elapsed();

    let mut file = fs::File::create(format!("{}/result.txt", models))?;
    file.write_all(format!("cv_score: {:?}\n\ntime used: {:?}", valid_acc, duration).as_bytes())?;

    loss_g.draw(format!("{}/loss.png", img))?;

    graph::draw_2hist(
        [valid_acc, train_acc],
        "Validation/Training Accuracy",
        ("Iterations", "Validataion/Training Accuracy"),
        format!("{}/acc.png", img),
    )?;

    graph::draw_confustion(matrix_vec, format!("{}/confusion_matrix.png", img))?;

    Ok(())
}
\end{minted}
\noindent model.rs
\begin{minted}[fontsize=\footnotesize, bgcolor=bg]{rust}
    use crate::activator;

    #[derive(Debug)]
    pub struct Layer {
        pub inputs: Vec<f64>,
        pub outputs: Vec<f64>, // need to save this for backward pass
        pub w: Vec<Vec<f64>>,
        pub b: Vec<f64>,
        pub grads: Vec<Vec<f64>>,
        pub w_prev_changes: Vec<Vec<f64>>,
        pub local_grads: Vec<f64>,
        pub b_prev_changes: Vec<f64>,
        pub act: activator::ActivationContainer,
    }
    
    impl Layer {
        pub fn new(
            input_features: u64,
            output_features: u64,
            bias: f64,
            act: activator::ActivationContainer,
        ) -> Layer {
            // initialize weights matrix
            let mut weights: Vec<Vec<f64>> = vec![];
            let mut inputs: Vec<f64> = vec![];
            let mut outputs: Vec<f64> = vec![];
            let mut grads: Vec<Vec<f64>> = vec![];
            let mut local_grads: Vec<f64> = vec![];
            let mut w_prev_changes: Vec<Vec<f64>> = vec![];
            let mut b_prev_changes: Vec<f64> = vec![];
            let mut b: Vec<f64> = vec![];
    
            for _ in 0..output_features {
                outputs.push(0.0);
                local_grads.push(0.0);
                b_prev_changes.push(0.0);
                b.push(bias);
    
                let mut w: Vec<f64> = vec![];
                let mut g: Vec<f64> = vec![];
                for _ in 0..input_features {
                    if (inputs.len() as u64) < input_features {
                        inputs.push(0.0);
                    }
                    g.push(0.0);
                    // random both positive and negative weight
                    w.push(2f64 * rand::random::<f64>() - 1f64);
                }
                weights.push(w);
                grads.push(g.clone());
                w_prev_changes.push(g);
            }
            Layer {
                inputs,
                outputs,
                w: weights,
                b,
                grads,
                w_prev_changes,
                local_grads,
                b_prev_changes,
                act,
            }
        }
    
        pub fn forward(&mut self, inputs: &Vec<f64>) -> Vec<f64> {
            if inputs.len() != self.inputs.len() {
                panic!("forward: input size is wrong");
            }
            let mut result: Vec<f64> = vec![];
            for j in 0..self.outputs.len() {
                let mut sum: f64 = 0.0;
                // loop through input and add w*x + b to sum
                for i in 0..inputs.len() {
                    sum += self.w[j][i] * inputs[i];
                }
                sum += self.b[j];
                self.outputs[j] = sum;
                result.push((self.act.func)(sum));
            }
            self.inputs = inputs.clone();
            result
        }
    
        pub fn update(&mut self, lr: f64, momentum: f64) {
            for j in 0..self.w.len() {
                let delta_b = lr * self.local_grads[j] + momentum * self.b_prev_changes[j];
                self.b[j] -= delta_b; // update each neuron bias
                self.b_prev_changes[j] = delta_b;
                for i in 0..self.w[j].len() {
                    // update each weights
                    let delta_w = lr * self.grads[j][i] + momentum * self.w_prev_changes[j][i];
                    self.w[j][i] -= delta_w;
                    self.w_prev_changes[j][i] = delta_w;
                }
            }
        }
    
        pub fn zero_grad(&mut self) {
            for j in 0..self.outputs.len() {
                self.local_grads[j] = 0.0;
                for i in 0..self.grads[j].len() {
                    self.grads[j][i] = 0.0;
                }
            }
        }
    }
    
    #[derive(Debug)]
    pub struct Net {
        pub layers: Vec<Layer>,
    }
    
    impl Net {
        pub fn from_layers(layers: Vec<Layer>) -> Net {
            Net { layers }
        }
    
        pub fn new(architecture: Vec<u64>) -> Net {
            let mut layers: Vec<Layer> = vec![];
            for i in 1..architecture.len() {
                layers.push(Layer::new(
                    architecture[i - 1],
                    architecture[i],
                    1f64,
                    activator::sigmoid(),
                ))
            }
            Net { layers }
        }
    
        pub fn zero_grad(&mut self) {
            for l in 0..self.layers.len() {
                self.layers[l].zero_grad();
            }
        }
    
        pub fn forward(&mut self, input: &Vec<f64>) -> Vec<f64> {
            let mut result = self.layers[0].forward(input);
            for l in 1..self.layers.len() {
                result = self.layers[l].forward(&result);
            }
            result
        }
    
        pub fn update(&mut self, lr: f64, momentum: f64) {
            for l in 0..self.layers.len() {
                self.layers[l].update(lr, momentum);
            }
        }
    }
    
    #[cfg(test)]
    mod tests {
        use super::*;
    
        #[test]
        fn test_linear_new() {
            let linear = Layer::new(2, 3, 1.0, activator::linear());
            assert_eq!(linear.outputs.len(), 3);
            assert_eq!(linear.inputs.len(), 2);
    
            assert_eq!(linear.w.len(), 3);
            assert_eq!(linear.w[0].len(), 2);
            assert_eq!(linear.b.len(), 3);
    
            assert_eq!(linear.grads.len(), 3);
            assert_eq!(linear.w_prev_changes.len(), 3);
            assert_eq!(linear.grads[0].len(), 2);
            assert_eq!(linear.w_prev_changes[0].len(), 2);
            assert_eq!(linear.local_grads.len(), 3);
            assert_eq!(linear.b_prev_changes.len(), 3);
        }
    
        #[test]
        fn test_linear_forward1() {
            let mut linear = Layer::new(2, 1, 1.0, activator::sigmoid());
    
            for j in 0..linear.w.len() {
                for i in 0..linear.w[j].len() {
                    linear.w[j][i] = 1.0;
                }
            }
    
            assert_eq!(linear.forward(&vec![1.0, 1.0])[0], 0.9525741268224334);
            assert_eq!(linear.outputs[0], 3.0);
        }
    
        #[test]
        fn test_linear_forward2() {
            let mut linear = Layer::new(2, 2, 1.0, activator::sigmoid());
    
            for j in 0..linear.w.len() {
                for i in 0..linear.w[j].len() {
                    linear.w[j][i] = (j as f64) + 1.0;
                }
            }
            let result = linear.forward(&vec![0.0, 1.0]);
            assert_eq!(linear.outputs[0], 2.0);
            assert_eq!(linear.outputs[1], 3.0);
            assert_eq!(result[0], 0.8807970779778823);
            assert_eq!(result[1], 0.9525741268224334);
        }
    }   
\end{minted}
\noindent activator.rs
\begin{minted}[fontsize=\footnotesize, bgcolor=bg]{rust}
#[derive(Debug)]
pub struct ActivationContainer {
    pub func: fn(f64) -> f64,
    pub der: fn(f64) -> f64,
    pub name: String,
}

pub fn sigmoid() -> ActivationContainer {
    fn func(input: f64) -> f64 {
        1.0 / (1.0 + (-input).exp())
    }
    fn der(input: f64) -> f64 {
        func(input) * (1.0 - func(input))
    }
    ActivationContainer {
        func,
        der,
        name: "sigmoid".to_string(),
    }
}

pub fn relu() -> ActivationContainer {
    fn func(input: f64) -> f64 {
        return f64::max(0.0, input);
    }
    fn der(input: f64) -> f64 {
        if input > 0.0 {
            return 1.0;
        } else {
            return 0.0;
        }
    }
    ActivationContainer {
        func,
        der,
        name: "relu".to_string(),
    }
}

pub fn linear() -> ActivationContainer {
    fn func(input: f64) -> f64 {
        input
    }
    fn der(_input: f64) -> f64 {
        1.0
    }
    ActivationContainer {
        func,
        der,
        name: "linear".to_string(),
    }
}

#[cfg(test)]
mod tests {
    use super::*;

    #[test]
    fn test_sigmoid() {
        let act = sigmoid();

        assert_eq!((act.func)(1.0), 0.7310585786300048792512);
        assert_eq!((act.func)(-1.0), 0.2689414213699951207488);
        assert_eq!((act.func)(0.0), 0.5);
        assert_eq!((act.der)(1.0), 0.1966119332414818525374);
        assert_eq!((act.der)(-1.0), 0.1966119332414818525374);
        assert_eq!((act.der)(0.0), 0.25);
    }

    #[test]
    fn test_relu() {
        let act = relu();

        assert_eq!((act.func)(-1.0), 0.0);
        assert_eq!((act.func)(20.0), 20.0);
        assert_eq!((act.der)(-1.0), 0.0);
        assert_eq!((act.der)(20.0), 1.0);
    }
}

\end{minted}
\noindent loss.rs
\begin{minted}[fontsize=\footnotesize, bgcolor=bg]{rust}    
use crate::model;

pub struct Loss {
    outputs: Vec<f64>,
    desired: Vec<f64>,
    pub func: fn(f64, f64) -> f64,
    pub der: fn(f64, f64) -> f64,
}

impl Loss {
    /// Mean Squared Error
    pub fn mse() -> Loss {
        fn func(output: f64, desired: f64) -> f64 {
            0.5 * (output - desired).powi(2)
        }
        fn der(output: f64, desired: f64) -> f64 {
            output - desired
        }

        Loss {
            outputs: vec![],
            desired: vec![],
            func,
            der,
        }
    }

    /// Binary Cross Entropy
    pub fn bce() -> Loss {
        fn func(output: f64, desired: f64) -> f64 {
            -desired * output.ln() + (1.0 - desired) * (1.0 - output).ln()
        }
        fn der(output: f64, desired: f64) -> f64 {
            -(desired / output - (1.0 - desired) / (1.0 - output))
        }

        Loss {
            outputs: vec![],
            desired: vec![],
            func,
            der,
        }
    }

    pub fn criterion(&mut self, outputs: &Vec<f64>, desired: &Vec<f64>) -> f64 {
        if outputs.len() != desired.len() {
            panic!("outputs size is not equal to desired size");
        }

        let mut loss = 0.0;
        for i in 0..outputs.len() {
            loss += (self.func)(outputs[i], desired[i]);
        }
        self.outputs = outputs.clone();
        self.desired = desired.clone();
        loss
    }

    pub fn backward(&self, layers: &mut Vec<model::Layer>) {
        for l in (0..layers.len()).rev() {
            // output layer
            if l == layers.len() - 1 {
                for j in 0..layers[l].outputs.len() {
                    // compute grads
                    let local_grad = (self.der)(self.outputs[j], self.desired[j])
                        * (layers[l].act.der)(layers[l].outputs[j]);

                    layers[l].local_grads[j] = local_grad;

                    // set grads for each weight
                    for k in 0..(layers[l - 1].outputs.len()) {
                        layers[l].grads[j][k] =
                            (layers[l - 1].act.func)(layers[l - 1].outputs[k]) * local_grad;
                    }
                }
                continue;
            }
            // hidden layer
            for j in 0..layers[l].outputs.len() {
                // calculate local_grad based on previous local_grad
                let mut local_grad = 0f64;
                for i in 0..layers[l + 1].w.len() {
                    for k in 0..layers[l + 1].w[i].len() {
                        local_grad += layers[l + 1].w[i][k] * layers[l + 1].local_grads[i];
                    }
                }
                local_grad = (layers[l].act.der)(layers[l].outputs[j]) * local_grad;
                layers[l].local_grads[j] = local_grad;

                // set grads for each weight
                if l == 0 {
                    for k in 0..layers[l].inputs.len() {
                        layers[l].grads[j][k] = layers[l].inputs[k] * local_grad;
                    }
                } else {
                    for k in 0..layers[l - 1].outputs.len() {
                        layers[l].grads[j][k] =
                            (layers[l - 1].act.func)(layers[l - 1].outputs[k]) * local_grad;
                    }
                }
            }
        }
    }
}

#[cfg(test)]
mod tests {
    use super::*;

    #[test]
    fn test_mse_func() {
        assert_eq!((Loss::mse().func)(2.0, 1.0), 0.5);
        assert_eq!((Loss::mse().func)(5.0, 0.0), 12.5);
    }

    #[test]
    fn test_mse_der() {
        assert_eq!((Loss::mse().der)(2.0, 1.0), 1.0);
        assert_eq!((Loss::mse().der)(5.0, 0.0), 5.0);
    }

    #[test]
    fn test_mse() {
        let mut loss = Loss::mse();

        let l = loss.criterion(&vec![2.0, 1.0, 0.0], &vec![0.0, 1.0, 2.0]);
        assert_eq!(l, 4.0);

        loss.criterion(
            &vec![34.0, 37.0, 44.0, 47.0, 48.0],
            &vec![37.0, 40.0, 46.0, 44.0, 46.0],
        );
        assert_eq!(l, 4.0);
    }

    #[test]
    fn test_bce_func() {
        println!("{}", (Loss::bce().func)(0.9, 0.0));
        println!("{}", (Loss::bce().func)(0.9, 1.0));
    }
}
\end{minted}
\noindent utills/mod.rs
\begin{minted}[fontsize=\footnotesize, bgcolor=bg]{rust}        
pub mod data;
pub mod graph;
pub mod io;
\end{minted}
\noindent utills/data.rs
\begin{minted}[fontsize=\footnotesize, bgcolor=bg]{rust}        
use super::io::read_lines;
use rand::prelude::SliceRandom;
use serde::Deserialize;
use std::error::Error;

#[derive(Debug, Clone)]
pub struct Data {
    pub inputs: Vec<f64>,
    pub labels: Vec<f64>,
}
#[derive(Clone)]
pub struct DataSet {
    datas: Vec<Data>,
}

impl DataSet {
    pub fn new(datas: Vec<Data>) -> DataSet {
        DataSet { datas }
    }

    pub fn cross_valid_set(&self, percent: f64) -> Vec<(DataSet, DataSet)> {
        if percent < 0.0 && percent > 1.0 {
            panic!("argument percent must be in range [0, 1]")
        }
        let k = (percent * (self.datas.len() as f64)).ceil() as usize; // fold size
        let n = (self.datas.len() as f64 / k as f64).ceil() as usize; // number of folds
        let datas = self.get_shuffled().clone(); // shuffled data before slicing it
        let mut set: Vec<(DataSet, DataSet)> = vec![];

        let mut curr: usize = 0;
        for _ in 0..n {
            let r_pt: usize = if curr + k > datas.len() {
                datas.len()
            } else {
                curr + k
            };

            let validation_set: Vec<Data> = datas[curr..r_pt].to_vec();
            let training_set: Vec<Data> = if curr > 0 {
                let mut temp = datas[0..curr].to_vec();
                temp.append(&mut datas[r_pt..datas.len()].to_vec());
                temp
            } else {
                datas[r_pt..datas.len()].to_vec()
            };

            set.push((DataSet::new(training_set), DataSet::new(validation_set)));
            curr += k
        }
        set
    }

    pub fn data_points(&self) -> Vec<f64> {
        let mut data_points: Vec<f64> = vec![];
        for mut dt in self.datas.clone() {
            data_points.append(&mut dt.inputs);
            data_points.append(&mut dt.labels);
        }
        data_points
    }

    pub fn max(&self) -> f64 {
        self.data_points()
            .iter()
            .fold(f64::NAN, |max, &v| v.max(max))
    }

    pub fn min(&self) -> f64 {
        self.data_points()
            .iter()
            .fold(f64::NAN, |min, &v| v.min(min))
    }

    pub fn std(&self) -> f64 {
        let mean = self.mean();
        let data_points = self.data_points();
        let n = data_points.len() as f64;
        data_points
            .iter()
            .fold(0.0f64, |sum, &val| sum + (val - mean).powi(2) / n)
            .sqrt()
    }

    pub fn mean(&self) -> f64 {
        let data_points = self.data_points();
        let n = data_points.len() as f64;
        data_points.iter().fold(0.0f64, |mean, &val| mean + val / n)
    }

    pub fn len(&self) -> usize {
        self.datas.len()
    }

    pub fn get_datas(&self) -> Vec<Data> {
        self.datas.clone()
    }

    pub fn get_shuffled(&self) -> Vec<Data> {
        let mut shuffled_datas = self.datas.clone();
        shuffled_datas.shuffle(&mut rand::thread_rng());
        shuffled_datas
    }
}

pub fn minmax_norm(dataset: &DataSet, min: f64, max: f64) -> DataSet {
    let datas: Vec<Data> = dataset
        .get_datas()
        .into_iter()
        .map(|dt| {
            let inputs: Vec<f64> = dt.inputs.iter().map(|x| (x - min) / (max - min)).collect();
            let labels: Vec<f64> = dt.labels.iter().map(|x| (x - min) / (max - min)).collect();
            Data { inputs, labels }
        })
        .collect();
    DataSet::new(datas)
}

pub fn standardization(dataset: &DataSet, mean: f64, std: f64) -> DataSet {
    let datas: Vec<Data> = dataset
        .get_datas()
        .into_iter()
        .map(|dt| {
            let inputs: Vec<f64> = dt.inputs.iter().map(|x| (x - mean) / std).collect();
            let labels: Vec<f64> = dt.labels.iter().map(|x| (x - mean) / std).collect();
            Data { inputs, labels }
        })
        .collect();
    DataSet::new(datas)
}

pub fn un_standardization(value: f64, mean: f64, std: f64) -> f64 {
    value * std + mean
}

pub fn xor_dataset() -> DataSet {
    let inputs = vec![[0.0, 0.0], [0.0, 1.0], [1.0, 0.0], [1.0, 1.0]];
    let labels = vec![[0.0], [1.0], [1.0], [0.0]];
    let mut datas: Vec<Data> = vec![];
    for i in 0..4 {
        datas.push(Data {
            inputs: inputs[i].to_vec(),
            labels: labels[i].to_vec(),
        });
    }

    DataSet::new(datas)
}

pub fn flood_dataset() -> Result<DataSet, Box<dyn Error>> {
    #[derive(Deserialize)]
    struct Record {
        s1_t3: f64,
        s1_t2: f64,
        s1_t1: f64,
        s1_t0: f64,
        s2_t3: f64,
        s2_t2: f64,
        s2_t1: f64,
        s2_t0: f64,
        t7: f64,
    }

    let mut datas: Vec<Data> = vec![];
    let mut reader = csv::Reader::from_path("data/flood_dataset.csv")?;
    for record in reader.deserialize() {
        let record: Record = record?;
        let mut inputs: Vec<f64> = vec![];
        // station 1
        inputs.push(record.s1_t3);
        inputs.push(record.s1_t2);
        inputs.push(record.s1_t1);
        inputs.push(record.s1_t0);
        // station 2
        inputs.push(record.s2_t3);
        inputs.push(record.s2_t2);
        inputs.push(record.s2_t1);
        inputs.push(record.s2_t0);

        let labels: Vec<f64> = vec![f64::from(record.t7)];
        datas.push(Data { inputs, labels });
    }
    Ok(DataSet::new(datas))
}

pub fn cross_dataset() -> Result<DataSet, Box<dyn Error>> {
    let mut datas: Vec<Data> = vec![];
    let mut lines = read_lines("data/cross.pat")?;
    while let (Some(_), Some(Ok(l1)), Some(Ok(l2))) = (lines.next(), lines.next(), lines.next()) {
        let mut inputs: Vec<f64> = vec![];
        let mut labels: Vec<f64> = vec![];
        for w in l1.split(" ") {
            let v: f64 = w.parse().unwrap();
            inputs.push(v);
        }
        for w in l2.split(" ") {
            let v: f64 = w.parse().unwrap();
            // class 1 0 -> 1
            // class 0 1 -> 0
            labels.push(v);
            break;
        }
        datas.push(Data { inputs, labels });
    }
    Ok(DataSet::new(datas))
}
\end{minted}
\noindent utills/graph.rs
\begin{minted}[fontsize=\footnotesize, bgcolor=bg]{rust}      
use plotters::coord::Shift;
use plotters::prelude::*;
use std::error::Error;

pub struct LossGraph {
    loss: Vec<Vec<f64>>,
    valid_loss: Vec<Vec<f64>>,
}

impl LossGraph {
    pub fn new() -> LossGraph {
        let loss: Vec<Vec<f64>> = vec![];
        let valid_loss: Vec<Vec<f64>> = vec![];
        LossGraph { loss, valid_loss }
    }

    pub fn add_loss(&mut self, training: Vec<f64>, validation: Vec<f64>) {
        self.loss.push(training);
        self.valid_loss.push(validation);
    }
    /// Draw training loss and validation loss at each epoch (x_vec)
    pub fn draw_loss(
        &self,
        idx: u32,
        root: &DrawingArea<BitMapBackend, Shift>,
        loss_vec: &Vec<f64>,
        valid_loss_vec: &Vec<f64>,
        max_loss: f64
    ) -> Result<(), Box<dyn Error>> {
        let min_loss1 = loss_vec.iter().fold(f64::NAN, |min, &val| val.min(min));
        let min_loss2 = valid_loss_vec
            .iter()
            .fold(f64::NAN, |min, &val| val.min(min));
        let min_loss = if min_loss1.min(min_loss2) > 0.0 {
            0.0
        } else {
            min_loss1.min(min_loss2)
        };

        let mut chart = ChartBuilder::on(&root)
            .caption(
                format!("Loss {}", idx),
                ("Hack", 44, FontStyle::Bold).into_font(),
            )
            .margin(20)
            .x_label_area_size(50)
            .y_label_area_size(60)
            .build_cartesian_2d(0..loss_vec.len(), min_loss..max_loss)?;

        chart
            .configure_mesh()
            .y_desc("Loss")
            .x_desc("Epochs")
            .axis_desc_style(("Hack", 20))
            .draw()?;

        chart.draw_series(LineSeries::new(
            loss_vec.iter().enumerate().map(|(i, x)| (i + 1, *x)),
            &RED,
        ))?;

        chart.draw_series(LineSeries::new(
            valid_loss_vec.iter().enumerate().map(|(i, x)| (i + 1, *x)),
            &BLUE,
        ))?;

        root.present()?;
        Ok(())
    }

    pub fn max_loss(&self) -> f64 {
        f64::max(
            self.loss.iter().fold(f64::NAN, |max, vec| {
                let max_loss = vec.iter().fold(f64::NAN, |max, &val| val.max(max));
                f64::max(max_loss, max)
            }),
            self.valid_loss.iter().fold(f64::NAN, |max, vec| {
                let max_loss = vec.iter().fold(f64::NAN, |max, &val| val.max(max));
                f64::max(max_loss, max)
            })
        )
    }

    pub fn draw(&self, path: String) -> Result<(), Box<dyn Error>> {
        let root = BitMapBackend::new(&path, (2000, 1000)).into_drawing_area();
        root.fill(&WHITE)?;
        // hardcode for 10 iteraions
        let drawing_areas = root.split_evenly((2, 5));

        let mut loss_iter = self.loss.iter();
        let mut valid_loss_iter = self.valid_loss.iter();
        let max_loss = self.max_loss();
        for (drawing_area, idx) in drawing_areas.iter().zip(1..) {
            if let (Some(loss_vec), Some(valid_loss_vec)) =
                (loss_iter.next(), valid_loss_iter.next())
            {
                self.draw_loss(idx, drawing_area, loss_vec, valid_loss_vec, max_loss)?;
            }
        }
        Ok(())
    }
}

/// Draw histogram of given datas
pub fn draw_histogram(
    datas: Vec<f64>,
    title: &str,
    axes_desc: (&str, &str),
    path: String,
) -> Result<(), Box<dyn Error>> {
    let n = datas.len();
    let max_y = datas.iter().fold(f64::NAN, |max, &val| val.max(max));
    let mean = datas
        .iter()
        .fold(0.0f64, |mean, &val| mean + val / n as f64);

    let root = BitMapBackend::new(&path, (1024, 768)).into_drawing_area();
    root.fill(&WHITE)?;

    let mut chart = ChartBuilder::on(&root)
        .caption(title, ("Hack", 44, FontStyle::Bold).into_font())
        .margin(20)
        .x_label_area_size(50)
        .y_label_area_size(60)
        .build_cartesian_2d((1..n).into_segmented(), 0.0..max_y)?
        .set_secondary_coord(1..n, 0.0..max_y);

    chart
        .configure_mesh()
        .disable_x_mesh()
        .y_desc(axes_desc.1)
        .x_desc(axes_desc.0)
        .axis_desc_style(("Hack", 20))
        .draw()?;

    let hist = Histogram::vertical(&chart)
        .style(RED.mix(0.5).filled())
        .margin(10)
        .data(datas.iter().enumerate().map(|(i, x)| (i + 1, *x)));

    chart.draw_series(hist)?;

    chart
        .draw_secondary_series(LineSeries::new(
            datas.iter().enumerate().map(|(i, _)| (i + 1, mean)),
            BLUE.filled().stroke_width(2),
        ))?
        .label(format!("mean: {:.3}", mean))
        .legend(|(x, y)| PathElement::new(vec![(x, y), (x + 20, y)], &BLUE));

    chart
        .configure_series_labels()
        .label_font(("Hack", 14).into_font())
        .background_style(&WHITE)
        .border_style(&BLACK)
        .draw()?;

    root.present()?;
    Ok(())
}

pub fn draw_2hist(
    datas: [Vec<f64>; 2],
    title: &str,
    axes_desc: (&str, &str),
    path: String,
) -> Result<(), Box<dyn Error>> {
    let n = datas.iter().fold(0f64, |max, l| max.max(l.len() as f64));
    let max_y = datas.iter().fold(0f64, |max, l| {
        max.max(l.iter().fold(f64::NAN, |v_max, &v| v.max(v_max)))
    });
    let mean: Vec<f64> = datas
        .iter()
        .map(|l| {
            l.iter()
                .fold(0f64, |mean, &val| mean + val / l.len() as f64)
        })
        .collect();

    let root = BitMapBackend::new(&path, (1024, 768)).into_drawing_area();
    root.fill(&WHITE)?;

    let mut chart = ChartBuilder::on(&root)
        .caption(title, ("Hack", 44, FontStyle::Bold).into_font())
        .margin(20)
        .x_label_area_size(50)
        .y_label_area_size(60)
        .build_cartesian_2d((1..n as u32).into_segmented(), 0.0..max_y)?
        .set_secondary_coord(0.0..n, 0.0..max_y);

    chart
        .configure_mesh()
        .disable_x_mesh()
        .y_desc(axes_desc.1)
        .x_desc(axes_desc.0)
        .axis_desc_style(("Hack", 20))
        .draw()?;

    let a = datas[0].iter().zip(0..).map(|(y, x)| {
        Rectangle::new(
            [(x as f64 + 0.1, *y), (x as f64 + 0.5, 0f64)],
            Into::<ShapeStyle>::into(&RED.mix(0.5)).filled(),
        )
    });

    let b = datas[1].iter().zip(0..).map(|(y, x)| {
        Rectangle::new(
            [(x as f64 + 0.5, *y), (x as f64 + 0.9, 0f64)],
            Into::<ShapeStyle>::into(&BLUE.mix(0.5)).filled(),
        )
    });

    chart.draw_secondary_series(a)?;
    chart.draw_secondary_series(b)?;

    let v: Vec<usize> = (0..(n + 1.0) as usize).collect();
    chart
        .draw_secondary_series(LineSeries::new(
            v.iter().map(|i| (*i as f64, mean[0])),
            RED.filled().stroke_width(2),
        ))?
        .label(format!("mean: {:.5}", mean[0]))
        .legend(|(x, y)| PathElement::new(vec![(x, y), (x + 20, y)], &RED));

    chart
        .draw_secondary_series(LineSeries::new(
            v.iter().map(|i| (*i as f64, mean[1])),
            BLUE.filled().stroke_width(2),
        ))?
        .label(format!("mean: {:.5}", mean[1]))
        .legend(|(x, y)| PathElement::new(vec![(x, y), (x + 20, y)], &BLUE));

    chart
        .configure_series_labels()
        .position(SeriesLabelPosition::UpperRight)
        .label_font(("Hack", 14).into_font())
        .background_style(&WHITE)
        .border_style(&BLACK)
        .draw()?;

    root.present()?;
    Ok(())
}

/// Draw confusion matrix
pub fn draw_confustion(matrix_vec: Vec<[[i32; 2]; 2]>, path: String) -> Result<(), Box<dyn Error>> {
    let root = BitMapBackend::new(&path, (2000, 1000)).into_drawing_area();
    root.fill(&WHITE)?;
    // hardcode for 10 iteraions
    let drawing_areas = root.split_evenly((2, 5));
    let mut matrix_iter = matrix_vec.iter();

    for (drawing_area, idx) in drawing_areas.iter().zip(1..) {
        if let Some(matrix) = matrix_iter.next() {
            let mut chart = ChartBuilder::on(&drawing_area)
                .caption(
                    format!("Confusion Matrix {}", idx),
                    ("Hack", 32, FontStyle::Bold).into_font(),
                )
                .margin(20)
                .build_cartesian_2d(0i32..2i32, 2i32..0i32)?
                .set_secondary_coord(0f64..2f64, 2f64..0f64);

            chart
                .configure_mesh()
                .disable_axes()
                .max_light_lines(4)
                .disable_x_mesh()
                .disable_y_mesh()
                .label_style(("Hack", 20))
                .draw()?;

            chart.draw_series(
                matrix
                    .iter()
                    .zip(0..)
                    .map(|(l, y)| l.iter().zip(0..).map(move |(v, x)| (x, y, v)))
                    .flatten()
                    .map(|(x, y, v)| {
                        Rectangle::new(
                            [(x, y), (x + 1, y + 1)],
                            HSLColor(
                                240.0 / 360.0 - 240.0 / 360.0 * (*v as f64 / 20.0),
                                0.7,
                                0.1 + 0.4 * *v as f64 / 20.0,
                            )
                            .filled(),
                        )
                    }),
            )?;

            chart.draw_secondary_series(
                matrix
                    .iter()
                    .zip(0..)
                    .map(|(l, y)| l.iter().zip(0..).map(move |(v, x)| (x, y, v)))
                    .flatten()
                    .map(|(x, y, v)| {
                        let text: String = if x == 0 && y == 0 {
                            format!["TP:{}", v]
                        } else if x == 1 && y == 0 {
                            format!["FP:{}", v]
                        } else if x == 0 && y == 1 {
                            format!["FN:{}", v]
                        } else {
                            format!["TN:{}", v]
                        };

                        Text::new(
                            text,
                            ((2.0 * x as f64 + 0.7) / 2.0, (2.0 * y as f64 + 1.0) / 2.0),
                            "Hack".into_font().resize(30.0).color(&WHITE),
                        )
                    }),
            )?;
        }
    }

    root.present()?;
    Ok(())
}
\end{minted}
\noindent utills/io.rs
\begin{minted}[fontsize=\footnotesize, bgcolor=bg]{rust}      
use crate::activator;
use crate::model;
use serde_json::{json, to_writer_pretty, Value};
use std::error::Error;
use std::fs::create_dir;
use std::fs::File;
use std::io::Read;
use std::io::{self, BufRead};
use std::path::Path;

pub fn save(layers: &Vec<model::Layer>, path: String) -> Result<(), Box<dyn Error>> {
    let mut json: Vec<Value> = vec![];

    for l in layers {
        json.push(json!({
            "inputs": l.inputs.len(),
            "outputs": l.outputs.len(),
            "w": l.w,
            "b": l.b,
            "act": l.act.name
        }));
    }
    let result = json!(json);
    let file = File::create(path)?;
    to_writer_pretty(&file, &result)?;
    Ok(())
}

pub fn read_lines<P>(filename: P) -> io::Result<io::Lines<io::BufReader<File>>>
where
    P: AsRef<Path>,
{
    let file = File::open(filename)?;
    Ok(io::BufReader::new(file).lines())
}

pub fn read_file<P>(filename: P) -> Result<String, Box<dyn Error>>
where
    P: AsRef<Path>,
{
    let mut file = File::open(filename)?;
    let mut contents = String::new();
    file.read_to_string(&mut contents)?;
    Ok(contents)
}

pub fn load<P>(filename: P) -> Result<model::Net, Box<dyn Error>>
where
    P: AsRef<Path>,
{
    let contents = read_file(filename)?;

    let json: Value = serde_json::from_str(&contents)?;
    let mut layers: Vec<model::Layer> = vec![];

    for l in json.as_array().unwrap() {
        // default layer activation is simeple linear f(x) = x
        let mut layer = model::Layer::new(
            l["inputs"].as_u64().unwrap(),
            l["outputs"].as_u64().unwrap(),
            0.0,
            activator::linear(),
        );
        // setting activation function
        if l["act"] == "sigmoid" {
            layer.act = activator::sigmoid();
        }

        // setting weights and bias
        let w = l["w"].as_array().unwrap();
        let b = l["b"].as_array().unwrap();
        for j in 0..w.len() {
            layer.b[j] = b[j].as_f64().unwrap();
            let w_j = w[j].as_array().unwrap();
            for i in 0..w_j.len() {
                layer.w[j][i] = w_j[i].as_f64().unwrap();
            }
        }

        layers.push(layer);
    }

    Ok(model::Net::from_layers(layers))
}

/// Check if specify folder exists in models and img folder, if not create it
///
/// Return models path and img path
pub fn check_dir(folder: &str) -> Result<(String, String), Box<dyn Error>> {
    let models_path = format!("models/{}", folder);
    if !Path::new(&models_path).exists() {
        create_dir(&models_path)?;
    }
    let img_path = format!("report/images/{}", folder);
    if !Path::new(&img_path).exists() {
        create_dir(&img_path)?;
    }
    Ok((models_path, img_path))
}
\end{minted}